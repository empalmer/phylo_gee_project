\documentclass[10pt]{article}
\usepackage[utf8]{inputenc}
\usepackage[english]{babel}
\usepackage{mathrsfs}
\usepackage{graphicx}
\usepackage[euler]{textgreek}
\usepackage{fancyhdr} % For header
\usepackage{mdframed} % for boxes
\usepackage{geometry} % For margins
\usepackage{amsthm} % For theorems
\usepackage{amsmath,amsfonts} % For math
 \geometry{
 a4paper,
 left=.75in,
 right=.75in,
 top=1in,
 }
 \newcommand{\bb}{\mathbb}
 \usepackage{sectsty}
\subsectionfont{\large\underline}
\fancyhf{}
\renewcommand\thesection{}
\renewcommand\thesubsection{}
\fancyhead[L]{\textbf{\leftmark\rightmark}}
\fancyhead[R]{\textbf{Spring 2020 PhD research}}

\rfoot{Page \thepage}
\pagestyle{fancy}

\theoremstyle{definition}
\newtheorem*{definition}{Definition}
\newtheorem*{example}{Example}
\newtheorem*{theorem}{Theorem}
\newtheorem*{lemma}{Lemma}
\newtheorem*{corollary}{Corrolary}
\setlength{\parskip}{1em}
\setlength{\parindent}{0pt}
\begin{document}


\section{Week 1}

TODO

\begin{itemize}
  \item Broad goal: Fix code:
  \item Write up where problems are happening
  \item Start using less filtered dataset. Also try other covariate?
\end{itemize}



\subsection{Data filtering}
Only use day 32.

Prevalence threshold. Only include taxa with more than 1 read in at least 30\% of samples.
This results in 39 taxa and 68 samples. The percent of the data is 46.3. 

Then calculate absolute abundance.

The current covariate being used is

\end{document}
