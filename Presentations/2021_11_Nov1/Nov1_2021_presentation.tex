\documentclass{beamer}

\mode<presentation>{

%\usetheme{default}
%\usetheme{AnnArbor}
%\usetheme{Antibes}
%\usetheme{Bergen}
%\usetheme{Berkeley}
%\usetheme{Berlin}
%\usetheme{Boadilla}
%\usetheme{CambridgeUS}
%\usetheme{Copenhagen}
%\usetheme{Darmstadt}
%\usetheme{Dresden}
%\usetheme{Frankfurt}
%\usetheme{Goettingen}
%\usetheme{Hannover}
%\usetheme{Ilmenau}
%\usetheme{JuanLesPins}
%\usetheme{Luebeck}
\usetheme{Madrid}
%\usetheme{Malmoe}
%\usetheme{Marburg}
%\usetheme{Montpellier}
%\usetheme{PaloAlto}
%\usetheme{Pittsburgh}
%\usetheme{Rochester}
%\usetheme{Singapore}
%\usetheme{Szeged}
%\usetheme{Warsaw}

% As well as themes, the Beamer class has a number of color themes
% for any slide theme. Uncomment each of these in turn to see how it
% changes the colors of your current slide theme.

%\usecolortheme{albatross}
%\usecolortheme{beaver}
%\usecolortheme{beetle}
%\usecolortheme{crane}
%\usecolortheme{dolphin}
%\usecolortheme{dove}
%\usecolortheme{fly}
%\usecolortheme{lily}
%\usecolortheme{orchid}
%\usecolortheme{rose}
%\usecolortheme{seagull}
%\usecolortheme{seahorse}
%\usecolortheme{whale}
%\usecolortheme{wolverine}

%\setbeamertemplate{footline} % To remove the footer line in all slides uncomment this line
%\setbeamertemplate{footline}[page number] % To replace the footer line in all slides with a simple slide count uncomment this line

%\setbeamertemplate{navigation symbols}{} % To remove the navigation symbols from the bottom of all slides uncomment this line
}
\usepackage{bm}
\usepackage{amsmath, amsthm, amsfonts}
\usepackage{graphicx} % Allows including images
\usepackage{booktabs} % Allows the use of \toprule, \midrule and \bottomrule in tables

%----------------------------------------------------------------------------------------
%	TITLE PAGE
%----------------------------------------------------------------------------------------
\title[]{Research Project Updates}
% \title[Phylogenetic GEEs]{Phylogenetic difference inspired GEE Correlation structure}  % The short title appears at the bottom of every slide, the full title is only on the title page

\author{Emily Palmer} % Your name
\institute[OSU] % Your institution as it will appear on the bottom of every slide, may be shorthand to save space
{
Oregon State University \\ % Your institution for the title page
\medskip
\textit{palmerem@oregonstate.edu} % Your email address
}
\date{\today} % Date, can be changed to a custom date

\begin{document}

\begin{frame}
\titlepage % Print the title page as the first slide
\end{frame}




%----------------------------------------------------------------------------------------
%	PRESENTATION SLIDES
%----------------------------------------------------------------------------------------
\begin{frame}
\frametitle{Goals for this presentation}
\begin{itemize}
  \item Introduce current project
  \item Get feedback on approach.
  \item Broad research question: Using GEEs, how can we specify a reasonable mean and variance function and working correlation matrix to model microbiome data
\end{itemize}
\end{frame}
%----------------------------------------------------------------------------------------
\begin{frame}
\frametitle{Ideas from Masters Project}
For my masters project I used the MLTC model using Generalized Estimating Equations (GEEs) for longitudinal data assumed to be taxonomically correlated
\begin{itemize}
  \item Assume that observations that share the same taxonomic group (at some level) have the same correlation based on the taxonomic tree. Combine with any distinct correlations from longitudinal measures.
  \item Global test for association between response (OTU counts/proportions) and  covariates (eg disease status).
  \item Two part model modeling presence/absence and log-transformed abundance separately
\end{itemize}
\end{frame}
%----------------------------------------------------------------------------------------


%----------------------------------------------------------------------------------------
\begin{frame}
\frametitle{Issues with previous model - Moving forward}
  \begin{itemize}
    \item Previous method is computationally infeasible
    \item Challenge of picking what taxa level to aggregate to: tradeoff of how many models to run/interpretability - arbitrary
    %TODO explain problems with interpretations better
    \item Compositionally is not considered - interpretations challenging
    \item Used taxonomy instead of phylogeny instead of - taxonomy may be too coarse
    \item The correlation matrix may not be positive semi-definite
  \end{itemize}
\end{frame}
%----------------------------------------------------------------------------------------

%----------------------------------------------------------------------------------------
\begin{frame}
\frametitle{GEE Framework}
\begin{itemize}
  \item Consider $i = 1, \ldots , n$ samples/individuals, each with $j = 1, \ldots , p$ OTU/taxa observations. The response vector for the $i$th subject is $\mathbf{Y}_i = (Y_{i1}, \ldots, Y_{ip})^T$. Denote $E(Y_{ij}) = \mu_{ij}$
  \item The collection of observations for a sample make up a cluster. Responses within a cluster are not independent, responses between clusters are independent.
  \item Consider covariates $\mathbf{X}_i = (X_{i1}, \ldots, X_{iq})$ for each sample. Usually covariates have the same value for all $p$ OTU observations in the $i$th subject.
  \item Use a link function between the covariates and mean response as $g(\mu_{ij}) = x_{ij}^T \beta$
  \item Variance is a function of the mean. $\text{Var}(Y_{ij}) = \phi a(\mu_{ij})$
\end{itemize}
\end{frame}

%----------------------------------------------------------------------------------------
\begin{frame}
\frametitle{GEE Equations}

\begin{itemize}
  \item $\beta$ is the solution to

  $$\sum_{i=1}^n \left(\frac{\partial  \boldsymbol\mu_i}{\partial \boldsymbol\beta  }\right)^T  \mathbf{V}_i^{-1}(\mathbf{Y}_i - \boldsymbol\mu_i) = 0 $$
  \item Where $\mathbf{V}_i = \mathbf{A}_i^{1/2}\mathbf{R}\mathbf{A}_i^{1/2}$
  \item   $A_i$ is a diagonal matrix whose entries are the variances
  \item $R$ is the \textbf{working correlation matrix}
\end{itemize}
\end{frame}
%----------------------------------------------------------------------------------------

%----------------------------------------------------------------------------------------
\begin{frame}
\frametitle{GEE Algorithm}
% FINISH TODO finish GEE algorithm
Initialize $\beta,\phi, R$
\begin{itemize}
  \item Update $\mu$, $e_{ik}$, $\phi$ based on current values of $\beta$
  \item Update $R$, $R^{-1}$ using $\phi$, $e_{ik}$
  \item Update $\beta$ iteratively using estimating equations and current values of $R, \mu, \phi, e_{ik}$
\end{itemize}
Repeat until convergence
\end{frame}
%----------------------------------------------------------------------------------------

%----------------------------------------------------------------------------------------
\begin{frame}
\frametitle{What to modify}
\begin{itemize}
  \item How to specify the working correlation matrix?
  \item What leads to a lack of independence?
  \begin{itemize}
    \item Phylogenetic correlation?
    \item Compositionality
    \item (Longitudinal repeated measures - incorporate eventually)
  \end{itemize}
  \item What mean and variance function to use?
  \begin{itemize}
    \item Previously a two-part model - binomial + log transformed gaussian
    \item Now - Dirichlet
  \end{itemize}
\end{itemize}
\end{frame}
%----------------------------------------------------------------------------------------

%----------------------------------------------------------------------------------------
\begin{frame}
\frametitle{Evolutionary trait model}
\begin{itemize}
  \item We borrow the idea of the evolutionary trait model (Martins and Hansen 1997) used in Microbiome data models (Xiao et al 2018)
  \item From a phylogenetic tree, create matrix $\mathbf{D}$ where $d_{ij}$ is the distance between OTU $i$ and $j$
  \item Use patristic distance - length of the shortest path
  \item Correlation between OTU $j$ and $k$ is
  $$Cor(Y_{ij}, Y_{ik}) = C_{jk}(\rho) = e^{-2\rho d_{jk}}$$ Where $\rho \in (0,\infty)$ and needs to be estimated.
If $\rho$ is small, $C_{jk}$ is close to 1 indicating high correlation. If $\rho$ is large, indicates no correlation.
  \item Interpretation of $\rho$: depth of the phylogenetic tree where groups are formed.
   % Hopefully some biological interpretation here?

  % Correlation matrix is positive definite.

\end{itemize}
\end{frame}
%----------------------------------------------------------------------------------------
%----------------------------------------------------------------------------------------
\begin{frame}
\frametitle{Potential issues}

\begin{itemize}
  \item Imposes a positive correlation requirement
  \item Assumes all OTUs that have the same distance are correlated the same
  \item Does not address compositionally
\end{itemize}

How to fix?

\end{frame}
%----------------------------------------------------------------------------------------
%----------------------------------------------------------------------------------------
\begin{frame}
\frametitle{Add a sign term}

\begin{itemize}
  \item One potential fix is to add a sign term to allow for negative correlations
  \item Now let
  $$\text{Corr}(Y_{ij},Y_{ik}) = c_{ij}e^{-2\rho d_{ij}}$$
  where $c_{ij} \in \{-1,1\}$

  \item How to set $c_{ij}$?


\end{itemize}
\end{frame}
%----------------------------------------------------------------------------------------


%----------------------------------------------------------------------------------------
\begin{frame}
\frametitle{Parameter estimation for $\rho$ and $c_{ij}$}
\begin{itemize}
  \item In the GEE algorithm, each iteration has a step to estimate $R$ equivalent to estimating $\rho$ and $c_{ij}$ using the current value of $\boldsymbol\beta$ in the iteration.

  \item One approach is to use OLS to find $\rho$ such that $\log e_{ij}e_{ik} = -2\rho d_{jk}$, where $e_{ij}$ is the Pearson residual. However, $e_{ij}$ can be negative. Instead take the absolute value: $\log |e_{ij}e_{ik}| = -2\rho d_{jk}$

  \item Another approach is to use the residuals from all subjects.

  $$\left| \frac{1}{n} \sum_{i=1}^{n} e_{ij}e_{ik} \right| = e^{-2\rho d_{jk}}$$

  \item Let $c_{jk} = \text{sign}\left(\sum_{i=1}^n e_{ij}e_{ik}\right)$

  \item Need to use constrained optimization to ensure that $\rho > 0$.

  % \item Is the estimation of $c_{ij}$ stable across iterations?
\end{itemize}

\end{frame}
%----------------------------------------------------------------------------------------


% Can we still have phylogenetic tree? with metagenomic data
% Called OTUs? Asvs? Species?
%----------------------------------------------------------------------------------------
\begin{frame}
\frametitle{Mean and variance function }
\begin{itemize}
  \item If the responses are proportions, we use the same mean and variance function as those from the Dirichlet$(\alpha_1, \ldots, \alpha_p)$ distribution
   $$\mu_{ij} = E(Y_{ij}) = \frac{\alpha_{ij}}{\alpha_{i0}}$$
  Where $\sum_{j=1}^p \alpha_{ij} = \alpha_{i0}$

  $$V(Y_{ij}) = a(\mu_{ij}) = \frac{\alpha_{ij}(\alpha_{i0} - \alpha_{ij})}{\alpha_{i0}^2(\alpha_{i0}+1)}$$
  % \item Dirichlet random variables have covariance:
  % $$Cov(Y_{ij},Y_{ik}) = - \frac{\alpha_j\alpha_k}{\alpha_0^2(\alpha_0 + 1)}$$
  \item Using the Dirichlet distribution we account for $\sum_{j=1}^p Y_{ij}= 1$
  \item Link using
  $$\log \alpha_{i} = X_i\beta$$
  % \item Are these alphas different for different samples? Or the same?


\end{itemize}
\end{frame}
%----------------------------------------------------------------------------------------
%----------------------------------------------------------------------------------------
\begin{frame}
\frametitle{Correlation idea from Dirichlet distribution}

Dirichlet random variables have a correlation that describes dependence from the constant sum constraint of the components of the cluster.

$$Corr(Y_{ij},Y_{ik})_{\text{Dir}} = - \frac{\alpha_{ij}\alpha_{ik}}{\alpha_{i0}^2(\alpha_{i0} + 1)}\frac{1}{\sqrt{V(Y_{ij})V(Y_{ik})}}$$

This will be negative.

\end{frame}
%----------------------------------------------------------------------------------------
%----------------------------------------------------------------------------------------
\begin{frame}
\frametitle{Compositional and Phylogenetic Correlations}
\begin{itemize}
  \item Model the correlations as a mixture of the phylogenetic correlation and compositional correlation. We will combine this covariance that arises from the Dirichlet distribution for compositional dependence.
  \item
  $$R = \omega R_{ETM} + (1 - \omega) R_{Dir}$$
  % \item This gives a mixture of a correlation that is required positive and a correlation that is required negative.
  Where $\omega \in [0,1]$
  \item Interpretation if compositionality or phylogenetic structure is the leading cause of correlation between responses.
  \item Is the sign term $c_{jk}$ still needed?
  \item We now also need to estimate $\omega$
  % This should be positive semidefinite because the sum is.
  % is this guaranteed to be a correlation matrix??
\end{itemize}
\end{frame}
%----------------------------------------------------------------------------------------

%----------------------------------------------------------------------------------------
\begin{frame}
\frametitle{Real data}
\begin{itemize}
  \item iHMP-IBD data
  \item Does metagenomic data generate a phylogenetic tree?
  \item Explore the data to see if this is a reasonable approach to model.
\end{itemize}
\end{frame}
%----------------------------------------------------------------------------------------


\begin{frame}
\Huge{\centerline{Thank you!}}
% TODO include references
\end{frame}

%----------------------------------------------------------------------------------------

\end{document}
