\documentclass{beamer}
\mode<presentation>
% The Beamer class comes with a number of default slide themes
% which change the colors and layouts of slides. Below this is a list
% of all the themes, uncomment each in turn to see what they look like.
  %\usetheme{default}
  %\usetheme{AnnArbor}
  %\usetheme{Antibes}
  %\usetheme{Bergen}
  %\usetheme{Berkeley}
  %\usetheme{Berlin}
  %\usetheme{Boadilla}
  %\usetheme{CambridgeUS}
  %\usetheme{Copenhagen}
  %\usetheme{Darmstadt}
  %\usetheme{Dresden}
  %\usetheme{Frankfurt}
  %\usetheme{Goettingen}
  %\usetheme{Hannover}
  %\usetheme{Ilmenau}
  %\usetheme{JuanLesPins}
  %\usetheme{Luebeck}
  \usetheme{Madrid}
  %\usetheme{Malmoe}
  %\usetheme{Marburg}
  %\usetheme{Montpellier}
  %\usetheme{PaloAlto}
  %\usetheme{Pittsburgh}
  %\usetheme{Rochester}
  %\usetheme{Singapore}
  %\usetheme{Szeged}
  %\usetheme{Warsaw}

  % As well as themes, the Beamer class has a number of color themes
  % for any slide theme. Uncomment each of these in turn to see how it
  % changes the colors of your current slide theme.

  %\usecolortheme{albatross}
  %\usecolortheme{beaver}
  %\usecolortheme{beetle}
  %\usecolortheme{crane}
  %\usecolortheme{dolphin}
  %\usecolortheme{dove}
  %\usecolortheme{fly}
  %\usecolortheme{lily}
  %\usecolortheme{orchid}
  %\usecolortheme{rose}
  %\usecolortheme{seagull}
  %\usecolortheme{seahorse}
  %\usecolortheme{whale}
  %\usecolortheme{wolverine}
  %\setbeamertemplate{footline} % To remove the footer line in all slides uncomment this line
  %\setbeamertemplate{footline}[page number] % To replace the footer line in all slides with a simple slide count uncomment this line
  %\setbeamertemplate{navigation symbols}{} % To remove the navigation symbols from the bottom of all slides uncomment this line

  \usepackage{graphicx} % Allows including images
  \usepackage{amsmath,amsthm,amsfonts,mathtools,color,amssymb}
  \usepackage{bm}
  \usepackage{booktabs} % Allows the use of \toprule, \midrule and \bottomrule in tables

  \title[Research Update]{Spring research update} % The short title appears at the bottom of every slide, the full title is only on the title page

  \author{Emily Palmer} % Your name
  \institute[OSU] % Your institution as it will appear on the bottom of every slide, may be shorthand to save space
  {Oregon State University \\ % Your institution for the title page
    \medskip
    \textit{palmerem@oregonstate.edu} % Your email address
  }
  \date{\today} % Date, can be changed to a custom date

  \begin{document}

  \begin{frame}
  \titlepage % Print the title page as the first slide
  \end{frame}

  \begin{frame}
  \frametitle{Overview} % Table of contents slide, comment this block out to remove it
  \tableofcontents % Throughout your presentation, if you choose to use \section{} and \subsection{} commands, these will automatically be printed on this slide as an overview of your presentation
  \end{frame}


%----------------------------------------------------------------------------------------
\begin{frame}
\frametitle{Goals}
\begin{itemize}
  \item Assume the mean and variance follow Dirichlet equations
  \item Assume the correlation structure between ASVs in a sample depends on compositionality and phylogenetic similarity
  \item Use GEEs to estimate regression parameters and covariance parameters
\end{itemize}
\end{frame}
%----------------------------------------------------------------------------------------

%----------------------------------------------------------------------------------------
\begin{frame}
\frametitle{Notation and setup}
\begin{itemize}
  \item Let $y_{ij}$ be the relative abundance of the $j$th ASV in the $i$th sample. $i = 1, \ldots , n, j = 1, \ldots , p$
  \item Assume that $E(y_{ij})  = \mu_{ij} = \frac{\alpha_{ij}}{\alpha_{i0}}$
  where $\alpha_{i0} = \sum_{j = 1}^p \alpha_{i0}$.

  and $\alpha_i$ are the parameters of if
  $\mathbf{y}_i \sim \text{Dirichlet}(\alpha_1, \ldots , \alpha_p)$

  \item Link function: Link covarites to $\alpha$'s:
  $$\log(\alpha_{ij}) = \mathbf{x_i}^T\boldsymbol\beta_j$$
  and
  $$\alpha_{ij} = e^{\mathbf{x_i}^T\boldsymbol\beta_j}$$
\end{itemize}
\end{frame}
%----------------------------------------------------------------------------------------


%----------------------------------------------------------------------------------------
\begin{frame}
\frametitle{Compositional Dirichlet Correlation}
\begin{itemize}
  \item Since ASVs are in relative abundances, we believe there will be a negative correlation arising from compositionality.
  \item Dirichlet correlation:
  for $j \neq k$
  \[ Cor(y_{ij}, y_{ik})  = - \frac{\alpha_{ij}\alpha_{ik}}{\alpha_{i0}^2(\alpha_{i0} + 1)\sqrt{V(y_{ij})V(y_{ik})}} \]
\end{itemize}
\end{frame}
%----------------------------------------------------------------------------------------


\begin{frame}
\frametitle{Evolutionary Trait Correlation}
\begin{itemize}
  \item We borrow the idea of the evolutionary trait model (Martins and Hansen 1997) used in Microbiome data models (Xiao et al 2018)
  \item From a phylogenetic tree, create matrix $\mathbf{D}$ where $d_{ij}$ is the distance between OTU $i$ and $j$.
  \item Use patristic distance - length of the shortest path.
  \item Correlation between OTU $j$ and $k$ is
  $$R_{i,ETM} = Cor(Y_{ij}, Y_{ik})  = e^{-2\rho d_{jk}}$$ Where $\rho \in (0,\infty)$ and needs to be estimated.
\item If $\rho$ is small, $C_{jk}$ is close to 1 indicating high correlation. If $\rho$ is large, indicates no correlation.
  \item Interpretation of $\rho$: depth of the phylogenetic tree where groups are formed.
\end{itemize}
\end{frame}


%----------------------------------------------------------------------------------------
\begin{frame}
\frametitle{Working Correlation matrix}
\begin{itemize}
  \item Use weighted sum of Dirichlet compositional correlation and evolutionary trait model correlation
  \[R = \omega R_{Dir} + (1-\omega)R_{ETM} \]
\end{itemize}
\end{frame}
%----------------------------------------------------------------------------------------

%----------------------------------------------------------------------------------------
\begin{frame}
\frametitle{GEE Equations}
The GEE equations are

\begin{align*}
  \sum_{i = 1}^n  \left(\frac{\partial  \boldsymbol\mu_i }{\partial \boldsymbol\beta }\right)^t\mathbf{V}_i^{-1}(\mathbf{Y_i} - \boldsymbol\mu_i) = 0
\end{align*}

\begin{itemize}
  \item Where $\boldsymbol V_i = \frac{1}{\phi} A_i^{\tfrac{1}{2}}R_iA_i^{\tfrac{1}{2}}$

  \item $A_i = V(Y_{ij})$

  \item $\phi = $

  \item \[ \left(\frac{\partial  \boldsymbol\mu_i }{\partial \boldsymbol\beta }\right)^t  = \frac{1}{\alpha_{i0}^2} (\alpha_{i0} \text{diag}(\alpha_i) - \alpha_i \alpha_i^t)  \otimes X_i\]


\end{itemize}



\end{frame}
%----------------------------------------------------------------------------------------

%----------------------------------------------------------------------------------------
\begin{frame}
\frametitle{Algorithm - $\rho, \omega, \phi$ step}

GEE algorithm goes between steps for estimating $\rho, \omega, $ and $\phi$ and step for estimating $\beta$.

\begin{itemize}
  \item $e_{ij} = y_{ij} - \mu_{ij}$
  \item $\phi = \frac{1}{\frac{1}{n*p - (p*q - 1)}\sum_{i = 1}^n \sum_{j = 1}^p e_{ij}^2}$
  \item $\omega, \rho$ minimize
  \[ \sum (\phi e_{ij}e_{ik} - [ \omega R_{jk,D} + (1 - \omega) e^{-2 \rho D_{jk}}])^2  \]
  suject to $0 \leq \omega \leq 1, \rho > 0$
  \item Given $\rho, \omega,$ working correlation $R$ is specified.
\end{itemize}

\end{frame}
%----------------------------------------------------------------------------------------

%----------------------------------------------------------------------------------------
\begin{frame}
\frametitle{Algorithm - $\beta$ step}

\[G =  \sum_{i = 1}^n  \left(\frac{\partial  \boldsymbol\mu_i }{\partial \boldsymbol\beta }\right)^t\mathbf{V}_i^{-1}(\mathbf{Y_i} - \boldsymbol\mu_i)\]
\[H = - \sum_{i=1}^n \left(\frac{\partial  \boldsymbol\mu_i }{\partial \boldsymbol\beta }\right)^t V_i ^{-1} \frac{\partial  \boldsymbol\mu_i }{\partial \boldsymbol\beta } + \lambda \]

\[\hat\beta^{(s +1)} = \hat\beta^{(s)} - \gamma H ^{-1} G \]

Where $0 < \gamma < 1$ is calculated by line search

\end{frame}
%----------------------------------------------------------------------------------------

%----------------------------------------------------------------------------------------
\begin{frame}
\frametitle{Real data (intro)}
a
\end{frame}
%---------------------------------------------------------

%----------------------------------------------------------------------------------------
\begin{frame}
\frametitle{Real Data - Results}
a
\end{frame}
%----------------------------------------------------------------------------------------

%----------------------------------------------------------------------------------------
\begin{frame}
\frametitle{Next steps and questions}
a
\end{frame}
%----------------------------------------------------------------------------------------



  \end{document}
