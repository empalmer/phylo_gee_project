\documentclass[10pt]{article}
\usepackage{sectsty}
\usepackage[utf8]{inputenc}
\usepackage[english]{babel}
\usepackage{mathrsfs}
\usepackage{graphicx}
\usepackage[euler]{textgreek}
\usepackage{fancyhdr} % For header
\usepackage{mdframed} % for boxes
\usepackage{geometry} % For margins
\usepackage{amsthm} % For theorems
\usepackage{amsmath,amsfonts} % For math
 \geometry{
 a4paper,
 left=.75in,
 right=.75in,
 top=1in,
 }
 \newcommand{\bb}{\mathbb}

% For bibliography
\usepackage[
backend=biber,
style=authoryear,
sorting=ynt
]{biblatex}
\addbibresource{research_notes_summer2022.bib}


% For checklist
\usepackage{enumitem,amssymb}
\newlist{todolist}{itemize}{2}
\setlist[todolist]{label=$\square$}
\usepackage{pifont}
\newcommand{\cmark}{\ding{51}}%
\newcommand{\xmark}{\ding{55}}%
\newcommand{\done}{\rlap{$\square$}{\raisebox{2pt}{\large\hspace{1pt}\cmark}}%
\hspace{-2.5pt}}

\subsectionfont{\large\underline}
\fancyhf{}
\renewcommand\thesection{}
\renewcommand\thesubsection{}
\fancyhead[L]{\textbf{\leftmark\rightmark}}
\fancyhead[R]{\textbf{Summer 2022 Notes}}
%\rhead{\textbf{Summer Notes}}
%\lhead{\textbf{Week}}
\rfoot{Page \thepage}
\pagestyle{fancy}

\theoremstyle{definition}
\newtheorem*{definition}{Definition}
\newtheorem*{example}{Example}
\newtheorem*{theorem}{Theorem}
\newtheorem*{lemma}{Lemma}
\newtheorem*{corollary}{Corrolary}
\setlength{\parskip}{1em}
\setlength{\parindent}{0pt}
\begin{document}

\section{June 16}



\subsection{To-Do}
\begin{todolist}
  \item Writeup current research.
  \item Create bibliography
  \item Make current document of all definitions
  \item Writeup dirichlet properties
  \item Citations
  \item Biological definitions
  \item GEE notation
  \item Writeup of identifiability issue
\end{todolist}






\newpage
\section{June 17 - Meeting}

This week I focused on taking a step back and writing up current research, make a place for all definitions used in one place.

Todo:
Literature search on dirichlet, look at link function, see if used logit,

Implement penalty first.

Look up logit idea, remember

Continue

Start career search

\section{Wednesday Jun 22}


\begin{todolist}
  \item Work on writeup
  \begin{itemize}
      \item Create bibliography
      \item Make current document of all definitions
      \item Writeup dirichlet properties
      \item Citations
      \item Biological definitions
      \item GEE notation
      \item Writeup of identifiability issue
  \end{itemize}
  \item Literature search on dirichlet logit link function
  \item Implemement penalty

\end{todolist}



\subsection{Literature search on dirichlet link }

\cite{douma2019a} Use the log link like we were under normal parametrization: $\log(\alpha_c) = \eta_c = X_c\beta_c$.

Alternatively,

\[E(p_c) = \mu_c \]

and $var[p_c] = \frac{\mu_c(1 - \mu_c)}{\phi+ 1}$ , where $\phi = \alpha_0$.

Then the link function for $\mu$ is the multinomial logit function.

Somewhat reccomends alternate parametrization. Then only $C-1$ values fitted. treat $C$ as baseline category

$\phi$ as the precision parameter, can model with log function.

So model both $\mu$ and $\phi$? seperately?

Zero inflation: offers different transformation
\[p^* = \frac{p(n-1) + \frac{1}{C}}{n} \]
Also reccomended by \cite{maier2014}
This compresses the data symmetrically around .5 so extreme values are more affected.

Reccomended R packages: DIRMULT, BRMS, DirichletReg

Notes zero-augmented Dirichlet regression \cite{tsagris2017}

Uses the logit link? or the one with a 1 on the bottom.

Could be interesting to look further into this method


\section{June 23}

\begin{todolist}
  \item Work on writeup for first pomofocus
  \item Add penalty.
\end{todolist}


Change in beta order wouldnt change the GEE equations, but wondering how current formulation outputs old formulation? Explore code more.


\subsection{Start on the penalty}

I dont quite understand the derivatives parts...

We want to heavily penalize

\begin{itemize}
  \item Add penalty $\lambda \sum_{j = 1}^p \beta_j^2$ to GEE equation
  \item Add penalty
\end{itemize}

\subsection{Meeting}:

Messed up on penalty, shoudl be matrix of 1s! Try this again.

Lambda larger,

Why does it have noise in only one clump?

Try larger sample size? Is it variation? Seems like.

Add diagonal back to hessian.

Try continuous X

Something with intercept.
TODO: Constraint has to be for each beta. Ie the intercept betas have to sum to 0, same for all other covariate betas.


Interpretability on alphas. Cant interpret on betas since


Start trying the logit link function. Will need to re-derive the matrix of partial derivatives.



\newpage
\section{June 27th}

\begin{todolist}
  \item[\done] First \& Second pomofocus on the writeup
  \begin{itemize}
    \item Some spell check and GEE description, working on bib files.
  \end{itemize}
  \item Third and pomofocus on literature review
  \begin{itemize}
    \item Reading \cite{luna2020}
    \item Also zotero custimization
  \end{itemize}
\end{todolist}


\subsection{Literature review: \cite{luna2020}}
Could be potential for Journal club.
Abstract:
Longitudinal negative binomial mixed effects model
Use hazard function
Joint model: Longitudinal and time to event data.

Method for determining how much changes in microbiome affect disease onset.
Introduction
One previous method to use smoothing splines to determine time intervals in which compositions are different between groups.
Prev methods only identify associations, not determine how much.

Methods:
Joint model: finds associations between time-dependent covariates and event times. Uses longitudinal submodel to model time dependent covariates, and then using those in a time to event model.

Longitudinal submodel uses read counts (non Gaussian, overdispersed)
Uses neg binom distribution to model overdispersed count data.

Model subject specific taxon abundances over time using negative binom linear mixed effects model. (log link)
Include offset of log of total sequence reads

Event submodel: use predicted relative abundances from mixed model into a hazard function.


\section{Tuesday June 28}

\begin{todolist}
  \item 1st Pomofocus: Read paper
  \begin{itemize}
    \item Read some of \cite{holmes2012}
  \end{itemize}
  \item 2nd Pomofocus: Writeup
  \item 3rd Pomofocus: State of research, organize, plan.
\end{todolist}

\subsection{Read paper Dirichlet Multinomial Mixtures \cite{holmes2012}}

Cluster into different groups, each group has its own Dirichlet mixture.

\subsection{Writeup progress}

Working more on GEE section:
Things still needed to be done include:

\begin{todolist}
  \item  modifying the partials derivation,
  \item Introduction!
  \item GEE algorithm
  \item More on identifiability
\end{todolist}


\subsection{Understanding notes from meeting}

Consider a discrete valued covariate $x$. This mimics what I expect commonly used data to be like, for example disease status or antibiotic usage.

Consider the simulation values I have (although slightly simpler). Let $n = 500$ for simulations, which seems to give a decently valued estimate.

Let $p = 10$, the number of ASVs, let $q = 2$, for one covariate and an intercept. $x_i \in (0,1)$, with a 50/50 split. so,

\[\mathbf{x} = (\overbrace{0 \cdots 0}^{250}, \overbrace{1 \cdots 1}^{250}) \]


Then, assume the true $\beta$ is

\begin{align*}
  \beta &= \begin{pmatrix} \beta_{11} = 0 \\ \vdots \\
\beta_{p1} = 0 \\
\beta_{12} = 1 \\ \vdots \\\beta_{p/2,2} = 1 \\\beta_{p/2 + 1,2} = -1\\ \vdots \\ \beta_{p2} = -1 \end{pmatrix}
\end{align*}

This follows our \textbf{constraint}:

\begin{align*}
  \sum_{j = 1}^p \beta_{jk} = 0 \quad \text{for all } k
\end{align*}

Then we have:

\begin{align*}
  \log(\alpha_{ij}) = x_i\beta \\
  \alpha_{ij} = e^{x_i}\beta\\
   \begin{pmatrix}\alpha_{i = 1, j = 1}\\\vdots \\
   \alpha_{i = 1, j = p/2+1}  \\ \vdots \\
   \alpha_{i = 251, j = 1} \\ \vdots \\ \alpha_{i = 251, j = p/2+1} \\ \vdots
 \end{pmatrix} &= \begin{pmatrix} e^{\beta_{11} + x \beta_{12}} = e^{0 + 0 \cdot 1} \\ \vdots \\ e^{0 + 0 \cdot -1} \\ \vdots \\ e^{0 + 1} \\ \vdots \\e^{0 - 1} \\ \vdots
\end{pmatrix}
\end{align*}

Thus $\alpha_{ij} = e^0 = 1$ for $i = 1, \ldots, 250, j = 1, \ldots , p$, and for $i > 250$, $\alpha_{ij} = e^1$ for $j \leq p/2$ and $\alpha_{ij}= e^{-1}$ for $j > p/2$

\begin{align*}
  \alpha_{ij} = \begin{cases}
  1   &   i \leq 250\\
  e^1   & i > 250, j \leq p/2\\
  e^{-1} & i > 250, j > p/2
  \end{cases}
\end{align*}


Idea: since there is this weird interpretation problem, maybe we should focus on correlation output and build networks??

\newpage
\printbibliography
\end{document}
