\documentclass[10pt]{article}
\usepackage[utf8]{inputenc}
\usepackage[english]{babel}
\usepackage{mathrsfs}
\usepackage{graphicx}
\usepackage[euler]{textgreek}
\usepackage{fancyhdr} % For header
\usepackage{mdframed} % for boxes
\usepackage{geometry} % For margins
\usepackage{amsthm} % For theorems
\usepackage{amsmath,amsfonts} % For math
\usepackage{mathtools}
 \geometry{
 a4paper,
 left=.75in,
 right=.75in,
 top=1in,
 }
 \newcommand{\bb}{\mathbb}
 \usepackage{sectsty}

% For bibliography
\usepackage[
backend=biber,
style=authoryear,
sorting=ynt
]{biblatex}
\addbibresource{Writeup.bib}

\subsectionfont{\large\underline}
\fancyhf{}
\renewcommand\thesection{}
\renewcommand\thesubsection{}
\fancyhead[L]{\textbf{\leftmark\rightmark}}
\fancyhead[R]{\textbf{Summer writeup of research}}

\rfoot{Page \thepage}
\pagestyle{fancy}

\theoremstyle{definition}
\newtheorem*{definition}{Definition}
\newtheorem*{example}{Example}
\newtheorem*{theorem}{Theorem}
\newtheorem*{lemma}{Lemma}
\newtheorem*{corollary}{Corollary}
\setlength{\parskip}{1em}
\setlength{\parindent}{0pt}
\begin{document}

\tableofcontents

\newpage

\section{Motivation}

Start writing what might be a good introduction for any paper or chapter.

Include citations!

\section{Definitions}

\begin{itemize}
  \item ASV
  \item OTU
  \item Zero-Inflation
  \item Phylogenetic tree
  \item Compositional
  \item Covariates/variables/ coefficients
\end{itemize}

\section{Notation}

\begin{itemize}
  \item Number of samples: $n$
  \item Sample index: $i: i = 1, \ldots, n$
  \item Number of ASVs: $k$
  \item ASV index: $j: j = 1, \ldots , k$
  \item Number of covariates: $q$. Note that $q$ does not include the intercept.
  \item Covariate index: $k: k = 1, \ldots , q$

  \item Responses: Transformed counts into relative abundances.

  Note that an alternative transformation to relative abundance might be good. %ADDCITE

   \begin{align*}
     y_{ij}\\
     \textbf{y}_{np \times 1} &= \begin{pmatrix} y_{i = 1, j = 1} \\\vdots \\ y_{i = 1, j = p}\\ \vdots \\ y_{i = n, j = 1} \\ \vdots  \\y_{i = n, j = p}\end{pmatrix}
   \end{align*}

   \item

   Design matrix $\mathbf{x}$ ('little x').

\begin{align*}
  \mathbf{x}_{n \times q} &= \begin{pmatrix}x_{i=1, k = 1} & \cdots & x_{i = 1, k = q}\\
  x_{i=2, k = 1} & \cdots & x_{i = 2, k = q}\\
\vdots & \ddots &  \vdots \\
x_{i = n, k = 1} & \cdots & x_{i = n, k = q}\end{pmatrix}
\end{align*}


    We don't have coefficient values measured on each ASV (i.e. no $j$ index), they are only measured on each sample. Thus $x_{ij} = x_{i}$ for all $j = 1, \ldots , p$. Then we convert to design matrix big X:

   \item Design matrix $X$:

   \begin{align*}
     X_{np \times pq} &= x \otimes I_p\\
     &= \begin{pmatrix} x_{i = 1, k = 1} & \overbrace{0 \cdots  0}^p &   & x_{i = 1, k = 2} & 0  \cdots  0 & \cdots &  x_{i = 1, k = q} & 0 \cdots 0  \\
    0 & x_{i = 1, k = 1}  & 0 \cdots 0 & 0 &  x_{i = 1, k = 2} & 0 \cdots 0 & 0 & \\
    \vdots \\
    & 0 \cdots 0 & x_{i = 1, k = 1} \\
     x_{i = 2, k = 1} & 0 & \cdots & x_{i = 2, k = 2} & \cdots & x_{i = 2, k = q} & 0 \\
   \vdots  \end{pmatrix}
   \end{align*}


   \item Parameter vector $\beta$ with entries $\beta_{jk}$

   \begin{align*}
     \boldsymbol\beta_{pq \times 1} = \begin{pmatrix}
     \beta_{j = 1, k = 1}\\
     \beta_{j = 2, k = 1} \\\vdots \\
     \beta_{j = p, k = 1} \\
     \beta_{j = 1, k = 2} \\ \vdots \\
     \beta_{j = 1, k = q} \\ \vdots \\
     \beta_{j = p, k = q}
   \end{pmatrix}
   \end{align*}


   \item Link:

   Link covariates to response.

   First assume that $\textbf{y}$ has the same mean and covariance structure as if $\textbf{y}_i \sim Dir(\alpha_{1p}, \ldots , \alpha_{ip})$ for all $i = 1, \ldots, n$.

   Then,

   \begin{align*}
     g(\alpha) = \log(\alpha) &= X \beta \\
     \log(\alpha_{ij}) &= x_{i} \beta_{j}
   \end{align*}

Where $\beta_j = (\beta_{j, k = 1}, \ldots \beta_{j, k = q})^t$, and $x_i = (x_{i1}, \ldots , x_{iq})$

This link function makes sense since $\alpha > 0$.

\end{itemize}





\section{Dirichlet ideas }

Assume $\eta_i = (\eta_{i1}, \ldots , \eta_{ip}) \sim Dir(\alpha_{i1}, \ldots , \alpha_{ip})$, and $\sum_{j=1}^p \alpha_{ij} = \alpha_{i0}$

\subsection{Dirichlet Expected value}

\[E(\eta_{ij}) = \frac{\alpha_{ij}}{\alpha_{i0}} \]


\subsection{Dirichlet variance }

\begin{align*}
  Var(\eta_{ij}) &= \frac{\alpha_{ij}(\alpha_{i0} - \alpha_{ij})}{\alpha_{i0}^2(\alpha_{i0} + 1)}
\end{align*}

Note: $0 \leq Var(\eta_{ij}) \leq 1$

\subsection{Dirichlet covariance }

For $i \neq j$

\begin{align*}
  Cov(\eta_{is}, \eta_{it}) &=  - \frac{\alpha_{is}\alpha_{it}}{\alpha_{i0}^2(\alpha_{i0} + 1)}
\end{align*}


Note this matrix is singular.
Note this quantity is always negative

\subsection{Dirichlet correlation }

For $i \neq j$

\begin{align*}
  Cor(\eta_{is}, \eta_{it}) &= \frac{Cov(\eta_{is}, \eta_{it})}{\sqrt{Var(\eta_{is})Var(\eta_{it})}}\\
  &= -  \frac{\alpha_{is}\alpha_{it}}{\alpha_{i0}^2(\alpha_{i0} + 1)} \cdot \sqrt{\frac{\alpha_{i0}^2(\alpha_{i0} + 1)}{\alpha_{is}(\alpha_{i0} - \alpha_{is})}}\sqrt{\frac{\alpha_{i0}^2(\alpha_{i0} + 1)}{\alpha_{it}(\alpha_{i0} - \alpha_{it})}}\\
  &= - \sqrt{\frac{\alpha_{is}\alpha_{it}}{(\alpha_{i0} - \alpha_{is})(\alpha_{i0} - \alpha_{it})}}
\end{align*}


Note this is always negative

\section{GEEs }

The collection of measurements taken on a single sample are assumed to be correlated. The entries in $\mathbf{y}_{i}$ for a given $i$ are correlated.

One such analysis method useful for correlated data are Generalized Estimating Equations. Originally proposed by


The GEE equations are

\begin{align*}
  \sum_{i = 1}^n  \left(\frac{\partial  \boldsymbol\mu_i }{\partial \boldsymbol\beta }\right)_{pq \times p}^t\mathbf{V}_{i_{p \times p}}^{-1}(\mathbf{Y_i} - \boldsymbol\mu_i)_{p \times 1} = 0
\end{align*}


\begin{itemize}
  \item $\boldsymbol V_i = A_i^{\tfrac{1}{2}}R_iA_i^{\tfrac{1}{2}}$: Workign covariance matrix
  \item $diag(A_i) = \sqrt{Var(\eta_{i})}$: Square root of dirichlet variance.
  \item $R_i$: working correlation matrix, see below, mixture of dirichlet and phylogenetic correaltion.
  \item $\left(\frac{\partial  \boldsymbol\mu_i }{\partial \boldsymbol\beta }\right)$ : Matrix of partial derivatives, defined below
  \item $Y_i$: response ASV proportions $0 < Y_i < 1$
  \item $\mu_i$: Expected mean: $E(Y_i) = \frac{\alpha_i}{\alpha_{i0}}$
\end{itemize}




 Derivation of $\left(\frac{\partial  \boldsymbol\mu_i }{\partial \boldsymbol\beta }\right)$

\begin{align*}
  \left(\frac{\partial  \boldsymbol\mu_i }{\partial \boldsymbol\beta}\right)^t &= \left(\frac{\partial  }{\partial \boldsymbol \beta} \frac{\boldsymbol\alpha_{i}}{\alpha_{i0}} \right)^t\\
  &= \left(\frac{\partial  }{\partial \boldsymbol \beta} \frac{e^{\mathbf{x}_i \boldsymbol\beta}}{\sum_{j = 1}^pe^{\mathbf{x_{ij}}\boldsymbol\beta}} \right)^t\\
  &= \begin{pmatrix}
        \frac{\partial }{\partial \boldsymbol\beta_1}\frac{e^{\mathbf{x}_i^T \boldsymbol\beta_1}}{\sum_{j = 1}^pe^{\mathbf{x_i}^T \boldsymbol\beta_j}} & \cdots & \frac{\partial  }{\partial \boldsymbol\beta_1}\frac{e^{\mathbf{x}_i^T \boldsymbol\beta_p}}{\sum_{j = 1}^pe^{\mathbf{x}_i^T \boldsymbol\beta_j}}\\
        \vdots & \ddots & \vdots \\
        \frac{\partial  }{\partial \boldsymbol\beta_p}\frac{e^{\mathbf{x}_i^T \boldsymbol\beta_1}}{\sum_{j = 1}^pe^{\mathbf{x}_i^T \boldsymbol\beta_j}} & \cdots & \frac{\partial  }{\partial \boldsymbol\beta_p}\frac{e^{\mathbf{x}_i^T \boldsymbol\beta_p}}{\sum_{j = 1}^pe^{\mathbf{x}_i^T \boldsymbol\beta_j}}
  \end{pmatrix}\\
  &= \frac{1}{\alpha_{i0}^2} \begin{pmatrix} x_1 \alpha_1 \alpha_0 - \alpha_1 x_1 \alpha_1 & - \alpha_2 x_1 \alpha_1 & \cdots & - \alpha_p x_1 \alpha_1 \\
x_2 \alpha_1 \alpha_0 - \alpha_1 x_2 \alpha_1 & - \alpha_2 x_2 \alpha_1 & \cdots & -\alpha_p x_2 \alpha_1 \\
\vdots & \vdots &  \ddots & \vdots \\
x_q\alpha_1\alpha_0 - \alpha_1 x_q\alpha_1 & - \alpha_2 x_q \alpha_1 &  \cdots  & \vdots  \\
- \alpha_1x_1\alpha_2 & x_1 \alpha_2\alpha_0 - \alpha_2 x_2\alpha_2 &   & \\
\vdots & \vdots & \ddots & \vdots \\
- \alpha_1x_q\alpha_p & \cdots  & & x_q\alpha_p\alpha_0 - \alpha_px_p\alpha_p
\end{pmatrix}_{pq \times p}\\
&= \frac{1}{a_{i0}} \begin{pmatrix}x_1 \alpha_1 & 0 & \cdots & 0 \\
x_2 \alpha_1 & 0 & \cdots & 0 \\
\vdots & \vdots & \ddots & \vdots \\
x_q\alpha_1 & 0 & \cdots & 0 \\
0 & x_1 \alpha_2 & \cdots & 0 \\
0 & x_2 \alpha_2 & \cdots & 0 \\
\vdots & \vdots & \ddots & \vdots \\
0 & 0 & \cdots & x_1 \alpha_p
\end{pmatrix} - \frac{1}{\alpha_{i0}^2}\begin{pmatrix} \alpha_1^2 x_1 & \alpha_1\alpha_2 x_1 & \cdots & \alpha_1\alpha_p x_1 \\
\alpha_1^2 x_2 &\alpha_1\alpha_2 x_2 & \cdots &  \alpha_1\alpha_p x_2\\
\vdots & \vdots &\ddots & \vdots \\
\alpha_1^2 x_q & \alpha_1\alpha_2 x_q & \cdots & \alpha_1 \alpha_p x_q\\
\alpha_1\alpha_2x_1 & \alpha_2^2  x_1 & \cdots & \alpha_2 \alpha_p x_1 \\
\alpha_1\alpha_2x_2 & \alpha_2^2  x_2 & \cdots & \alpha_2 \alpha_p x_2\\
\vdots & \vdots & \ddots &\vdots \\
\alpha_1\alpha_p x_q & \alpha_2\alpha_p x_q & \cdots & \alpha_p^2 x_q\end{pmatrix}\\
% &= \frac{1}{\alpha_{i0}} \begin{pmatrix}\alpha_1 \mathbf{x}_i & 0 & \\
% & \ddots & \\
% 0 & & \alpha_p \mathbf{x}_i\end{pmatrix}_{pq \times p} - \frac{1}{\alpha_{i0}^2} \boldsymbol\alpha \boldsymbol\alpha^t \otimes \mathbf{x}_i\\
&= \frac{1}{\alpha_{i0}}[I_p \otimes \mathbf{x}_i] diag(\boldsymbol\alpha_i) - \frac{1}{\alpha_{i0}^2} \boldsymbol\alpha_i \boldsymbol\alpha_i^t \otimes \mathbf{x}_i
% &= \frac{1}{\alpha_{i0}^2}[\alpha_0 \boldsymbol1_p \boldsymbol\alpha \otimes \mathbf{x}_i - \boldsymbol\alpha \boldsymbol\alpha^t \otimes \mathbf{x}_i]
%   &= \begin{pmatrix}
%         \frac{\partial}{\partial \boldsymbol\beta_{j = 1, k = 1}}\frac{e^{\mathbf{x}_{i1} \boldsymbol\beta}}{\sum_{j = 1}^p e^{\mathbf{x}_{ij} \boldsymbol\beta}} & \cdots & \frac{\partial}{\partial \boldsymbol\beta_{1,1}}
%         \frac{e^{\mathbf{x}_{ip} \boldsymbol\beta}}{\sum_{j = 1}^pe^{\mathbf{x}_{ij} \boldsymbol\beta}}\\
%         \frac{\partial}{\partial \boldsymbol\beta_{j = 1, k = 2}}\frac{e^{\mathbf{x}_{i1} \boldsymbol\beta}}{\sum_{j = 1}^p e^{\mathbf{x}_{ij} \boldsymbol\beta}} & \vdots & \vdots \\
%         \vdots & \vdots & \vdots \\
%         \frac{\partial  }{\partial \boldsymbol\beta_{j = p, k = q}}\frac{e^{\mathbf{x}_{i1} \boldsymbol\beta}}{\sum_{j = 1}^pe^{\mathbf{x}_{ij} \boldsymbol\beta}} & \cdots & \frac{\partial  }{\partial \boldsymbol\beta}\frac{e^{\mathbf{x}_{ip} \boldsymbol\beta_{p,q}}}{\sum_{j = 1}^pe^{\mathbf{x}_{ij} \boldsymbol\beta}}
%   \end{pmatrix}_{np \times p}\\
%   &= \frac{1}{\left(\sum_{j = 1}^pe^{\mathbf{x}_{ij} \boldsymbol\beta} \right)^2} \begin{pmatrix}
% x_{i1} \alpha_{i1} \alpha_{i0} - x_{iq}\alpha_{i1}\alpha_{i1} & \cdots & - x_{i1}\alpha_{i1}\alpha_{ip}\\
% \vdots & \vdots & \vdots \\
% - x_{i1}\alpha_{i1}\alpha_{i2} & x_{i2} \alpha_{i0}\alpha_{i2} - x_{i2}\alpha_{i2}\alpha_{i2} & \cdots \\
% \vdots & \ddots & \vdots\\
% - x_{i1} \alpha_{i1}\alpha_{ip} & \cdots & x_{iq}\alpha_{ip}\alpha_{i0} - x_{iq} \alpha_{ip}\alpha_{ip}\\
% \vdots & \ddots & \vdots
%   \end{pmatrix}\\
  % &= \frac{1}{\alpha_{i0}^2} \left(\alpha_{i0} \mathbf{x}_i \text{diag}(\boldsymbol\alpha_i) - \mathbf{x}_i \boldsymbol\alpha_i^T \boldsymbol\alpha_i \right)
\end{align*}

\section{GEE algorithm}

Let $G$ be the GEE equation, and $H$ be the hessian.




\section{Identifiability issues}

Note that we can write the mean:

\begin{align*}
  \mu_{ij} &= \frac{\alpha_{ij}}{\alpha_{i0}}\\
  &= \frac{e^{x_i\beta_j}}{\sum_{s = 1}^p e^{x_i\beta_s }}\\
  &= \frac{e^{x_i\beta_j}}{e^{x_i\beta_1} + \cdots + e^{x_i\beta_p}}\\
  &= \frac{e^{\gamma}}{e^{\gamma}}\frac{e^{x_i\beta_j}}{e^{x_i\beta_1} + \cdots + e^{x_i\beta_p}}\\
  &=
\end{align*}

Thus $\mu_{ij}$ is the same regardless of if $\beta = (\beta_1, \ldots, \beta_p)$ or $\beta = (\beta_1 + \gamma, \cdots , \beta_p + \gamma)$.

Thus the mean is non-identifiable.

Check if this is the same for the variance.

\newpage
\section{Penalty}

To solve this problem, we first attempt to add a penalty.

Penalty:

\begin{align*}
  \lambda \left(\sum_{j=1}^p \beta_p \right)^2 &= \lambda \beta^t 1 1^t \beta
\end{align*}

New GEE eqn
\begin{align*}
 G^* &= G + \frac{\partial \text{Penalty}}{\partial \beta} \\
 &= G + 2\lambda 1 1^t \beta
\end{align*}

 New Hessian:

 \begin{align*}
   H^* &= H + \frac{\partial ^2 \text{Penalty}}{\partial ^2 } \\
   &= H + 2 \lambda 1 1^t
 \end{align*}


\newpage
\printbibliography


\end{document}
