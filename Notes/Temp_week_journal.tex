\documentclass[10pt]{article}
\usepackage[utf8]{inputenc}
\usepackage[english]{babel}
\usepackage{mathrsfs}
\usepackage{graphicx}
\usepackage[euler]{textgreek}
\usepackage{fancyhdr} % For header
\usepackage{mdframed} % for boxes
\usepackage{geometry} % For margins
\usepackage{amsthm} % For theorems
\usepackage{amsmath,amsfonts} % For math
 \geometry{
 a4paper,
 left=.75in,
 right=.75in,
 top=1in,
 }
 \newcommand{\bb}{\mathbb}
 \usepackage{sectsty}
\subsectionfont{\large\underline}
\fancyhf{}
\renewcommand\thesection{}
\renewcommand\thesubsection{}
\fancyhead[L]{\textbf{\leftmark\rightmark}}
\fancyhead[R]{\textbf{Temp Journal }}
%\rhead{\textbf{Temp Journal }}
%\lhead{\textbf{Week}}
\rfoot{Page \thepage}
\pagestyle{fancy}

\theoremstyle{definition}
\newtheorem*{definition}{Definition}
\newtheorem*{example}{Example}
\newtheorem*{theorem}{Theorem}
\newtheorem*{lemma}{Lemma}
\newtheorem*{corollary}{Corrolary}
\setlength{\parskip}{1em}
\setlength{\parindent}{0pt}
\begin{document}

\section{Week 4}

Concern that problems are due to high dimensionality. Focus this week: Filter to one Genus/family/something and repeat.

Start thinking/writing about motivations for needing/wanting to model everything at once instead of running separately.

What dataset to use? Prevalence filter first?
Going to use the 10p filtered dataset and then pick a genus.

Also looking into R ways to resolve phylogenetic tree to genus level.
Potentially group clade? https://yulab-smu.top/treedata-book/chapter6.html

(Needed to add phylogenetic tree to saved rds. )


\subsection{Meeting notes}

Found bug. Ugh. Variance term was incorrect. Check things now.


Try independence,
Try fixed values.
Look at residual matrix (convert to matrix?)
Check that entries match.




\section{Week 5}

Found bug. (In variances). Also note that diriclet variances should be between 0 and 1.

Check with $\gamma = 1$.


Check with $\lambda$


Vocab
\begin{itemize}
  \item $\gamma$: step size
  \item $\rho$: phylogenetic distance scaling
  \item $\omega$: correlation weighting.
  \item $\beta$ estimated parameter values/ coefficients
  \item $\lambda$: diagonal scalar to subtract from Hessian.
\end{itemize}


Tuesday:
Noticed code was getting ugly and hard to navigate. Changed return values to initialize easier by just initilizeing one list that has sub lists inside it.

With:


\end{document}
