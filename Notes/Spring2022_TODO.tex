\documentclass[10pt]{article}
\usepackage[utf8]{inputenc}
\usepackage[english]{babel}
\usepackage{mathrsfs}
\usepackage{graphicx}
\usepackage[euler]{textgreek}
\usepackage{fancyhdr} % For header
\usepackage{mdframed} % for boxes
\usepackage{geometry} % For margins
\usepackage{amsthm} % For theorems
\usepackage{amsmath,amsfonts} % For math
 \geometry{
 a4paper,
 left=.75in,
 right=.75in,
 top=1in,
 }
 \newcommand{\bb}{\mathbb}
 \usepackage{sectsty}
\subsectionfont{\large\underline}
\fancyhf{}
\renewcommand\thesection{}
\renewcommand\thesubsection{}
\fancyhead[L]{\textbf{\leftmark\rightmark}}
\fancyhead[R]{\textbf{Research TODO}}

\rfoot{Page \thepage}
\pagestyle{fancy}

% For checklist
\usepackage{enumitem,amssymb}
\newlist{todolist}{itemize}{2}
\setlist[todolist]{label=$\square$}
\usepackage{pifont}
\newcommand{\cmark}{\ding{51}}%
\newcommand{\xmark}{\ding{55}}%
\newcommand{\done}{\rlap{$\square$}{\raisebox{2pt}{\large\hspace{1pt}\cmark}}%
\hspace{-2.5pt}}
\newcommand{\wontfix}{\rlap{$\square$}{\large\hspace{1pt}\xmark}}
\newcommand{\inprogress}{\rlap{$\square$}{\large\hspace{1pt}$\bigstar$}}
\newcommand{\focus}{\ding{222}}
\newcommand{\order}[1]{\rlap{$\square$}{\large\hspace{1pt}(#1)}}


\theoremstyle{definition}
\newtheorem*{definition}{Definition}
\newtheorem*{example}{Example}
\newtheorem*{theorem}{Theorem}
\newtheorem*{lemma}{Lemma}
\newtheorem*{corollary}{Corrolary}
\setlength{\parskip}{1em}
\setlength{\parindent}{0pt}
\begin{document}

\section{Week 1}

\begin{itemize}
  \item Broad goal: Fix code.
  \begin{todolist}
    \item[\done] Use less filtered dataset
    \begin{itemize}
      \item Now at 30\% with 39 taxa.
    \end{itemize}
    \item[\done] Make graphs of parameter values at each iteration
    \begin{itemize}
      \item Did a lot of graphs, histograms and changes of betas between loops. Need to do final after changes are made.
    \end{itemize}
    \item[\done] Try stepwise update
    \begin{itemize}
      \item Using .1 as a constant. Still doesnt work, so likely a more complicated approach would not be helpful.
    \end{itemize}
    \item[\done] Fix problem where R inverse and others not updating in beta loop
    \item[\done] Change R-inv function to depend on alpha instead of X and beta since already calculated.
    \item[\done] Flip order of beta and phi loops.
    \item[\done] Look at residuals in each iteration. (in phi step)
    \item Keep something constant through loop and/or try using identity for R.
    \item[\done] Try different dataset? Maybe simulated one from previous paper.
    \begin{itemize}
      \item Same troubles
    \end{itemize}
    \item Distance matrix argument to dist?
    \begin{itemize}
      \item Need both forms, so may or may not be useful
    \end{itemize}
    \item Check if phi is correct now, and see how is used
    \item[\done] Figure out large alpha problem
    \begin{itemize}
      \item Seems the + and one loop beta fixed this?
    \end{itemize}
    \item Try a different covariate?
    \item Figure out why geem code uses + instead of - on update
    \begin{itemize}
      \item I think + is correct. At least it converges.
    \end{itemize}
    \item Identify which covariate is causing the trouble.
    \item Try removing the single OTU that seems to be contributing to problem.
    \item Save everything to list correctly to look at later.
    \begin{itemize}
      \item Saving rho, omega, phi, beta, diff (abs), num iterations
      \item Also do residuals. Which residuals? Cross product ones? Squared ones? Standardized squared ones? Probably standardized but not squared. Should be distributed around zero. ?
    \end{itemize}
    \item[\done]Plot lists
    \begin{itemize}
      \item Plotted rhos, omegas, phis, diffs, not sure what else to plot.
    \end{itemize}
    \item[\done] Remove scalar update.
    \begin{itemize}
      \item Doesn't work. see journal
    \end{itemize}
    \item Write up algorithm for Yuan to go through? Ask to check.
    \item Final run through before meeting of graphs and numbers
    \item[\done] Move timing into function
  \end{todolist}
\end{itemize}

\section{Week 2}
Main tasks:
\begin{itemize}
  \item To-do from sheet
  \item Prepare presentation
\end{itemize}




\begin{todolist}
  \item[\done] Wednesday email research update.
  \item[\done] Include more taxa: more zero-inflation
  \begin{itemize}
    \item Not great results. see journal omega = .99, rho = .1
    \item runtime of 20 min.
  \end{itemize}
  \item[\done] Add computation time in model.
  \item Try different covariates (parasite burden/antibiotics/exposure)
  \item Think about genus level analysis/ phylogenetic tree for taxonomic units
  \item Choose $\alpha$ automatically (simple line search)
  \item Write up line-search steps and goals
  \item Make presentation
  \item Make sure subtraction in $\phi$ is correct
  \item Figure out why adding the minus sign changes the code.
  \item Understand how Hessian is calculated and expected values equal zero.
  \item Run on previous dataset from original paper.
  \item Rename file with algorithm
  \item Compare to independence GEE written already geeglm?
  \item Calculate sandwhich variance estimators for asymptotic significance tests.
  \item See if new shared tree that Tom had is the same?
  \item Add residuals to output and look at plots
  \item[\inprogress] Change initial value to something different?
  \begin{itemize}
    \item change initial rho to 10. Changes results (of at least ests of rho and omega)
    \item Try previous iterations rho and omega.
    \item See how beta estimates change
    \item Run code with initial value of 1 again to compare.
  \end{itemize}
  \item Read numerical optimization book.
  \item[\done] Try changing the minus sign on larger dataset.
  \begin{itemize}
    \item Either immediate convergence or infinite alpha values on 1st iteration.....
  \end{itemize}
  \item[\done] Set up automatic save outputs along with description
  \item[\done] Try with a medium sized dataset, like prevalence of .2?
  \item[\done] Save test datasets to files for easy loading
  \item[\done] Only include last beta output?
  \item Save convergence results of nls optim
\end{todolist}
\end{document}
